
\chapter{Einleitung und Intension}
\label{sec:EinleitungUndIntension}
Die Programmiersprache Java galt lange Zeit als langsame Programmiersprache und
auch derzeit hält sich das Gerücht weiterhin.
Mit der Version 1.1.6 änderte sich das durch die Implementierung des
Just-In-Time Compilers\cite{symantec:symantec}.
Mit weiteren Versionen wurde weiter an der Performance gearbeitet. So wurden in
der letzten Hauptversion Java 7 (Stand 12.08.2014) unter anderem parallele
Rechnungen\cite{goetz:IBM} und ein neuer Garbage Collector
eingeführt\cite{humble:infoq}.
Auch mit Java 8 wird weiterhin die Performance verbessert\cite{gallardo:oracle}.
So sollen Lambda Ausdrücke performanter sein als innere Klassen und sind mit Hauptargument für
die Umstellung und Nutzung von der neuen Java Version.
 
Diese Arbeit beschäftigt sich mit einigen neu eingeführten Funktionen in die
Programmiersprache Java der Version 8 im Vergleich zur Version 7, geht dabei
auf die dabei zu beobachtende Performance, jedoch auch auf die Komplexität und
somit dem erhöhten Nutzen ein.
 
Das erste Kapitel beschreibt die Grundlagen der Programmiersprache
Java, die allgemeine Funktionsweise und dient dem allgemeinen Verständnis der
Arbeit.
Im zweiten Kapitel wird auf die theoretischen Vor- und Nachteile der neuen
Funktionen eingegangen, welche im dritten Kapitel praktisch untersucht werden.
In der Zusammenfassung wird Bezug nehmend auf die Ergebnisse ein Vergleich mit
der Version 7 gezogen.
 
\chapter{Grundlagen}
\label{sec:Grundlagen}
Hier steht der erste Text.

\section{Objektorientierte Programmierung}
\label{sec:OO}
Die objektorientierte Programmierung ist ein fundamentaler Programmierstil, der
auf dem Prinzip basiert, dass der von der zu Programmierenden Sache betroffene
Bereich der Wirklichkeit als Modell in der Architektur widergespiegelt ist.
Dabei werden Konzepte wie Klassen, Objekte und Vererbung genutzt.
In der Objektorientierten Programmiersprache Java kann so zum Beispiel die
Klasse "`Buch"' den Rahmen zum Erzeugen des gleichnamigen Objektes geben. Dabei
kann das Objekt mehrere Eigenschaften besitzen und weiteren Klassen, die
wiederum Objekte erzeugen, besitzen. 

\section{Java Virtual Machine}
\label{sec:JVM}

\section{Just-In-Time-Compiler}
\label{sec:JIT}



\chapter{Theoretische Untersuchung neuer Funktionen}
\label{sec:Theorie}

\section{Lambda Ausdrücke}
\label{sec:Lambda}

\chapter{Praktische Untersuchung neuer Funktionen}
\label{sec:Praxis}

\section{Lambda Ausdrücke}
\label{sec:Lambda}

\chapter{Zusammenfassung}
\label{sec:Fazit}