
\chapter{Einleitung und Intension}
\label{sec:EinleitungUndIntension}
Die Programmiersprache Java galt lange Zeit als langsame Programmiersprache und
auch derzeit hält sich das Gerücht weiterhin.
Mit der Version 1.1.6 änderte sich das durch die Implementierung des
Just-In-Time
Compilers\footnote{\url{http://www.symantec.com/about/news/release/article.jsp?prid=19970407_03}}.
Mit weiteren Versionen wurde weiter an der Performance gearbeitet. So wurden in der letzten Hauptversion Java 7 (Stand 12.08.2014) unter anderem parallele Rechnungen\footnote{\url{http://www.ibm.com/developerworks/java/library/j-jtp03048/index.html}}
 und ein neuer Garbage Collector eingeführt\footnote{\url{http://www.infoq.com/news/2008/05/g1}}.
 Auch mit Java 8 wird weiterhin die Performance verbessert. So sollen Lambda
 Ausdrücke performanter sein als innere Klassen\footnote{\url{https://blogs.oracle.com/thejavatutorials/entry/learn_more_about_performance_and}} und sind mit Hauptargument für die Umstellung und Nutzung von der neuen Java
 Version.
 
 Diese Arbeit beschäftigt sich mit den neu eingeführten Funktionen in die
 Programmiersprache Java der Version 8 im Vergleich zur Version 7, geht dabei
 auf die dabei zu beobachtende Performance, jedoch auch auf die Komplexität und
 somit dem erhöhten Nutzen ein.
 
 Das erste Kapitel beschreibt die Grundlagen der Programmiersprache
 Java, die allgemeine Funktionsweise und dient dem allgemeinen Verständnis der
 Arbeit.
 Im zweiten Kapitel wird auf die theoretischen Vor- und Nachteile der neuen
 Funktionen eingegangen, welche im dritten Kapitel praktisch untersucht werden.
 In der Zusammenfassung wird Bezug nehmend auf die Ergebnisse ein Vergleich mit
 der Version 7 gezogen.
 
\chapter{Grundlagen}
\label{sec:Grundlagen}
Hier steht der erste Text.

\section{Objektorientierung}
\label{sec:OO}
Und hier wird es schon spezifischer.

\section{Java Virtual Machine}
\label{sec:JVM}

\section{Just-In-Time-Compiler}
\label{sec:JIT}

In unserem Netzwerk setzen wir auf \gls{AD}. Durch den Einsatz
eines \gls{AD} erreichen wir bei \gls{T}-Systemen, die mit einer
\gls{glos:AntwD} von \gls{CD} installiert wurden, die beste Standardisierung.
\gls{UH}

\chapter{Theoretische Untersuchung neuer Funktionen}
\label{sec:Theorie}

\section{Lambda Ausdrücke}
\label{sec:Lambda}

\chapter{Praktische Untersuchung neuer Funktionen}
\label{sec:Praxis}

\section{Lambda Ausdrücke}
\label{sec:Lambda}

\chapter{Zusammenfassung}
\label{sec:Fazit}