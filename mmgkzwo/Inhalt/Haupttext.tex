
\chapter{Einleitung und Intension}
\label{sec:EinleitungUndIntension}
Die Programmiersprache Java galt lange Zeit als langsame Programmiersprache und
auch derzeit hält sich das Gerücht weiterhin.
Mit der Version 1.1.6 änderte sich das durch die Implementierung des
Just-In-Time Compilers\cite{symantec:symantec}.
Mit weiteren Versionen wurde weiter an der Performance gearbeitet. So wurden in
der letzten Hauptversion Java 7 (Stand 12.08.2014) unter anderem parallele
Rechnungen\cite{goetz:IBM} und ein neuer Garbage Collector
eingeführt\cite{humble:infoq}.
Auch mit Java 8 wird weiterhin die Performance verbessert\cite{gallardo:oracle}.
So sollen Lambda Ausdrücke performanter sein als innere Klassen und sind mit Hauptargument für
die Umstellung und Nutzung von der neuen Java Version.
 
Diese Arbeit beschäftigt sich mit einigen neu eingeführten Funktionen in die
Programmiersprache Java der Version 8 im Vergleich zur Version 7, geht dabei
auf die dabei zu beobachtende Performance, jedoch auch auf die Komplexität und
somit dem erhöhten Nutzen ein.
 
Das erste Kapitel beschreibt die Grundlagen der Programmiersprache
Java, die allgemeine Funktionsweise und dient dem allgemeinen Verständnis der
Arbeit.
Im zweiten Kapitel wird auf die theoretischen Vor- und Nachteile der neuen
Funktionen eingegangen, welche im dritten Kapitel praktisch untersucht werden.
In der Zusammenfassung wird Bezug nehmend auf die Ergebnisse ein Vergleich mit
der Version 7 gezogen.
 
\chapter{Java Grundlagen}
\label{sec:Grundlagen}
Bei der Programmiersprache Java handelt es sich um eine objektorientierte
Programmiersprache, welche die Java Virtual Machine als Kernstück zur Ausführung
des Codes benutzt. Dabei spielen der Just-In-Time-Compiler, sowie der Garbage
Collector eine wichtige Rolle. Die grundlegenden Begriffe werden in den
Folgenden Kapiteln beschrieben.

\section{Objektorientierte Programmierung}
\label{sec:OO}
Die objektorientierte Programmierung ist ein fundamentaler Programmierstil, der
auf dem Prinzip basiert, dass der von der zu Programmierenden Sache betroffene
Bereich der Wirklichkeit als Modell in der Architektur widergespiegelt wird.
Dabei werden Konzepte wie Klassen, Objekte und Vererbung genutzt.
In der Objektorientierten Programmiersprache Java kann so zum Beispiel die
Klasse "`Buch"' den Rahmen zum Erzeugen des gleichnamigen Objektes geben. Dabei
kann das Objekt mehrere Eigenschaften und weiteren Klassen, die
wiederum Objekte erzeugen, besitzen. Ein Buchobjekt kann somit aus mehreren
Objekten "`Seite"' bestehen, welche wiederum Eigenschaften wie "`Seitenzahl"'
erhalten kann. Weiterhin können Subklassen Eigenschaften der Klasse "`Buch"'
erben und durch andere Attribute erweitert werden.
Vorteile sollen dabei sein: die Wiederverwendbarkeit von Code, die effektivere
Arbeit im Team durch Modularisierung und durch Abbildung der Wirklichkeit eine
schnellere Erlernbarkeit. Nachteile sollen im Gegensatz zu prozeduralen Sprachen
die geringere Laufzeiteffektivität und Energieeffizienz sein.

\section{Java Virtual Machine}
\label{sec:JVM}
Die Java Virtual Machine(kurz JVM) ist eine virtuelle Maschine, die
Java-Byte-Code ausführen kann. Sie ist der Grundstein jeder Java Applikation und
somit unabdingbar. Die JVM ist es, die Java seine Plattform- und Hardware
Unabhängigkeit gibt. Wie eine echte Rechenmaschine hat auch die JVM einen Befehlssatz und
manipuliert mehrere Speicheradressen zeitgleich. Sie verarbeitet nur ein
bestimmtes Binärformat der Java Dateien, `.class` Dateien. Ansonsten weiß die
JVM nichts von der Programmiersprache Java.

\section{Just-In-Time-Compiler}
\label{sec:JIT}
Der Just-In-Time-Compiler, welcher auch kurz JIT genannt wird, wird unter
anderem auch von der Java Virtual Machine benutzt. JIT Compiler wandeln 
Quelltext hauptsächlich zur Laufzeit in ausführbare Befehle um. Dieses Prinzip
bringt mehrere Vorteile, denn zum einen werden nur Programmteile kompiliert die
auch wirklich durchlaufen werden, was Ressourcen spart. Zum anderen entfallen
lange Startzeiten, da nur ein kleiner Teil Anfangs kompiliert wird. Bei früheren
Java Versionen stellte die mitunter lange Startzeit ein Problem dar. Hinzu
kommt, dass bereits kompilierte Teile des Programms im Cache gehalten werden,
was bedeutet, dass der Compiler diesen Teil nicht noch einmal durchlaufen muss.

\section{Garbage Collector}
\label{sec:GC}
Trotz des JIT Compilers ist Java eine speicherlastige Sprache, Abhilfe soll hier
durch den Garbage Collector geschaffen werden. Seine Funktion besteht darin,
nicht mehr benutzte Objekte aus dem Speicher aufzuspüren und diesen Speicher
wieder frei zu geben. Je nach Ausstattung der Systemhardware ist dies von Nöten,
da die JVM theoretisch bis zu 2\textasciicircum64 Adressen zur Verfügung hat.
Natürlich nur bei 64 Bit basierten Systemen und mit entsprechender Hardware. Beim Start kann 
jedoch meist festgelegt werden wieviel Speicher die JVM allokiert, also
für den Gebrauch reservieren und für anderen Applikationen sperren soll. Der
Garbage Collector sollte für den Benutzer kaum spürbar sein, er verrichtet seine Arbeit
normalerweise dann wenn die Hardwarelast gering genug ist. Als Ausnahme gilt
allerdings wenn der allokierte Speicher voll ist und mehr Speicher, zum Beispiel für das
erstellen eines großen Objektes benötigt wird, denn an dieser Stelle muss der
Garbage Collector auch bei hoher Last seine Arbeit verrichten.

\chapter{Theoretische Untersuchung neuer Funktionen}
\label{sec:Theorie}

\section{Lambda Ausdrücke}
\label{sec:Lambda}
Um Lambda Ausdrücke zu verstehen sollte man den grundsätzlichen Unterschied
zwischen funktionaler und imperativer Programmiersprachen kennen. Vereinfacht
formuliert: "`bei imperativen Herangehensweisen, welche unter anderem bei Java
vorkommt, wird im Detail beschrieben, was der Code macht und wie er
funktioniert.
Bei funktionalen Sprachen, wie Haskell, wird das Problem als Satz von
Funktionen formuliert, so dass die Reihenfolge der Berechnungen nicht angegeben
werden muss."' Das führt dazu, dass diese Reihenfolge nicht relevant
ist.
Folgende Vorteile werden dabei nutzbar. Mit Lambda Ausdrücken kann eine komplexe
Funktion beschrieben werden, welche aus mehreren Funktionen besteht und es muss erst
dann dafür gesorgt werden, dass alle Bestandteile zur Verfügung stehen wenn sie
benötigt werden. Das begünstigt Verschachtelung von Funktionen, begrenzt
allerdings auch die Leserlichkeit von eben diesen Funktionen. Im Speziellen ist
die Nutzung dieser Ausdrücke für Gruppen kritisch, sofern nicht alle
Mitglieder Wissen über funktionale Sprachen bzw.
Ansätze aufweisen.
Lambda Ausdrücke sind anonyme Funktionen, die also nur an der Stelle benutzt
werden können an der sie beschrieben sind. Diese Tatsache ist bei der Verwendung
von Lambda Ausdrücken wichtig, denn wenn diese Funktion
noch einmal gebraucht werden, dann sollten sie nicht als Lambda Funktion
beschrieben werden.
Lambda Ausdrücke haben die Vorteile, dass wenn man einmal geübt ist, jede Menge
Code Zeilen, innere Klassen, oder komplett neue Klassen eingespart werden können
und der Code trotzdem noch lesbar ist. Außerdem muss die JVM dabei weniger Daten
behandeln, als bei anderen Herangehensweisen. Letzteres resultiert daher, dass
sich Klassen bzw. Funktionen der Klassen gegenseitig Objekte hin und
herschieben, was bei komplexeren Problemen, bzw. Datenlastigen Objekte auch größere Dimensionen und damit erhöhte Rechenleistung mit sich bringt.

\section{Funktionale Interfaces}
\label{sec:FI}
Neben den Lambda-Expressions haben mit Java 8 außerdem funktionale Interfaces
Einzug erhalten. Jedes Interface welche nur eine Methode besitzt, wird als
funktionales Interface bezeichnet. Interfaces mit dieser Intention erhalten in
Java die Annotation @FunctionalInterface. Sollte das Interface nicht den
Rahmenbedingungen eines funktionalen Interface genügen wird der Compiler einen
Fehler erzeugen. Zu erwähnen sei noch, dass egal ob die Annotation vorhanden ist
oder nicht, der Compiler jedes Interface, welches die Bedingungen erfüllt,
als funktionales Interface behandeln wird.
Funktionale Interfaces sind in der Lage Lambda Ausdrücke zu verwenden.

\section{Default-Implementierung}
\label{sec:DI}
Bisher hatte man in Java-Interfaces keine Methoden Rümpfe, doch mit der
Einführung der default-Implementierung hat sich auch das geändert. Es sind
Standardimplementierungen für eine Methode eines Interfaces. Das hat zur Folge,
dass der Programmierer selbst entscheiden kann, ob er eine bestimmte Methoden
überschreibt um sie für ein spezielles Objekt effektiver zu gestalten.

\section{Iterable-Interface und die forEach-Methode}
\label{sec:IIufEM}
Das Iterable-Interface ist gedacht für alle Arten von Objekten die
"`durchlaufen"' werden können. Eine Neuerung in diesen Interfaces ist die
forEach-Methode. Diese hilft bei der deutlichen Trennung von Geschäftslogik des
Objekts und dem Code. An dieser Stelle sei gesagt, dass diese Änderung am Anfang
nur nach mehr Arbeit aussieht. In Verbindung mit Lambda Ausdrücken und
Funktionalen Interfaces können damit aber sehr effektive Programme entstehen. 


\chapter{Praktische Untersuchung neuer Funktionen}
\label{sec:Praxis}

\section{Lambda Ausdrücke}
\label{sec:Lambda}
Um zu veranschaulichen wie sich Lambda Ausdrücke auf den Programmierstil
auswirken, sieht man nachfolgend an einem Beispiel. Hier wird ein einfacher
Button beschrieben welcher die Aufgabe hat auf Knopfdruck eine Nachricht auf der
Java Console auszugeben.

\lstset{language=Java}  
\begin{lstlisting}
JButton but = new JButton("Push me!");
but.addActionListener(new ActionListener(){
	@override
	public void actionPerformed(ActionEvent e){
		System.out.println("Command: " 
		+ e.getActionCommand());
	}
});
\end{lstlisting}

In Java hat man wie oben ersichtlich bisher sogenannte Anonyme Innerclasses
verwendet, um den Knopf mit dem Event zu verknüpfen. Davon ausgehend das die
Funktion die dieser Knopf erfüllt an keiner anderen Stelle noch einmal so benutzt wird ist das vom Programmierstil her
sauber. Dennoch ist für eine so "`kleine"' Aufgabe viel Code von Nöten, dessen
Leserlichkeit durch seine Struktur begrenzt ist.

\pagebreak

\begin{lstlisting}
JButton but = new JButton("Push me!");
button.addActionListener(
	(ActionEvent e) -> System.out.plrintln("Command: " 
	+ e.getActionCommand())
);
\end{lstlisting}
Dieser Code beschreibt und verhält sich genau so wie der oben Aufgeführte. Der
einzige Unterschied ist das er mit einem Lambda Ausdruck realisiert wurde.
Ein Lambda Ausdruck ist in 3 Teile geteilt:
\begin{itemize}
	\item Parameter
	\item Pfeil-Operator
	\item Funktionskörper
\end{itemize}

Der Parameter ist in diesem Fall das "`ActionEvent e"' die Angabe des Types
ActionEvent hilft dabei Typfehler auszuschließen. In Java wird das benötigt, da
die Sprache nicht Typsicher ist. Der Pfeil-Operator ist die Verknüpfung zwischen
Parameter und Körper und wird durch "`->"' symbolisiert. Alles was nach dem
Pfeiloperator kommt ist der Funktionskörper, der seinerseits auch wieder einen
Lambda Ausdruck enthalten kann.

\section{Default-Implementierung}
\label{sec:DI}

Die neu eingeführte Default-Implementierung von Interfaces soll am folgenden
Beispiel veranschaulicht werden.

\begin{lstlisting}
public Interface Update {
	public void refreshValue(Integer int);
}
\end{lstlisting}

Dieses Interface wird von den folgenden beiden Klassen implementiert.

\begin{lstlisting}
public Class NeedsValueUpdate implements Update {
	@Override
	public void refresh(Integer int) {
		value = int;
	}
}
\end{lstlisting}

\begin{lstlisting}
public Class NeedsValueUpdateToo implements Update {
	@Override
	public void refresh(Integer int) {
		value = int;
	}
}
\end{lstlisting}

Für eine neue Anforderung soll das Interface nun erweitert werden. Bisher war es
notwendig die Methode in jeder Klasse, welche das Interface implementiert
hatte, ebenso zu implementieren. Bei mehreren Klassen bedeutet dies einen nicht
zu vernachlässigbaren Aufwand. Die Default-Implementierung lässt jetzt zu, dass
direkt im Interface eine Standardimplementierung festgelegt werden kann. Damit
ist es nicht mehr notwendig die Klassen anzupassen.

\begin{lstlisting}
public Interface Update {
	public void refreshValue(Integer int);
	
	default public void printValue(Integer int) {
		System.out.println("Value: " + int);
	}
}
\end{lstlisting}

Diese Funktion bringt jedoch auch Nachteile mit sich. Es besteht die
Möglichkeit, dass eine Klasse zwei Interfaces implementiert. Wenn diese
Interfaces nun zwei Standard-Implementierungen von Methoden mit gleichem Namen
beinhalten, tritt ein Konflikt auf.
\newline

\begin{lstlisting}
public Interface AnotherUpdate {
	default public void printValue(Integer int) {
		System.out.println("Other Value: " + int);
	}
}
\end{lstlisting}

Wenn dieses Interface nun von der Klasse NeedsValueUpdateToo ebenso
implementiert wird, dann muss manuell angegeben werden, welche
Standard-Implementierung genutzt werden soll, besonders wenn die Methode nicht
überschrieben wird.

\begin{lstlisting}
public Class NeedsValueUpdateToo implements Update, AnotherUpdate {
	@Override
	public void refresh(Integer int) {
		value = int;
	}
	
	@Override
	public void printValue(Integer int) {
		AnotherUpdate.super.printValue(int);
	}
}
\end{lstlisting}

\section{Funktionale Interfaces, Iterable-Interface und die forEach-Methode}
\label{sec:IIufEM}

Bis Java 8 wurde das Durchlaufen von Daten folgendermaßen durchgeführt:

\begin{lstlisting}
List<String> stringList = Arrays.asList("first", "second", "third");
for (String string : stringList) {
	System.out.println(string);
}
\end{lstlisting}

Mit der neuen forEach-Methode, die im Iterable-Interface dazugekommen ist, wird
die gleiche Funktion wie folgt aufgerufen:

\begin{lstlisting}
List<String> stringList = Arrays.asList("first", "second", "third");
stringList.forEach(new Consumer<String>() {
	public void accept(String string) {
		System.out.println(string);
	}
}
\end{lstlisting}

Diese Implementierung bringt eine zwar eine bessere Trennung von Iteration und
Verarbeitung der Elemente mit sich, ist jedoch länger, verschachtelter und
dadurch unübersichtlicher als die zuvor gezeigte. Abhilfe schaffen hier
Lambda-Expressions, die den Quelltext im Gegensatz zum ersten Beispiel verkürzen
und Übersichtlichkeit bewahren. Somit kann die Abbildung des Durchlaufens auf
eine Zeile verringert werden.
Die Lambda-Expression darf bei diesem Beispiel genutzt werden, da das Interface
Iterable ein FunctionalInterface ist.

Der unter Hilfenahme der oben beschriebenen Methoden resultierende Quelltext
wird im folgenden dargestellt.

\begin{lstlisting}
List<String> stringList = Arrays.asList("first", "second", "third");
stringList.forEach((String string) -> System.out.println(string));
\end{lstlisting}

\chapter{Zusammenfassung}
\label{sec:Fazit}